\documentclass{a0poster}
\usepackage[margin=0cm, paperwidth=90cm, paperheight=120cm]{geometry}
\usepackage{poster}

\begin{document}
\begin{center}
    \colorbox{nottblue!100}{%
        \begin{minipage}[t]{\textwidth} % Adjusted vertical alignment
            \vspace{0.8em} % Add vertical space
            \begin{center}
                {\fontsize{85pt}{85pt}\selectfont\textbf{\textcolor{white}{Equivalence between the second order steady state for the spin-boson model
and its quantum mean force Gibbs state}}}\\[1ex] % Increased font size
               \Large \textit{\textcolor{white}{ Prem Kumar }}\textsuperscript{\textcolor{white}{1,2}}, \textit{\textcolor{white}{\textbf{K. P. Athulya}}}\textsuperscript{\textcolor{white}{1,2}}, \textit{\textcolor{white}{Sibasish Ghosh}}\textsuperscript{\textcolor{white}{1,2}}\\
    \textit{\textsuperscript{\textcolor{white}{1}}\textcolor{white}{ Optics \& Quantum Information Group, The Institute of Mathematical Sciences, CIT Campus, Taramani, Chennai 600113, India}}\\

    \textit{\textsuperscript{\textcolor{white}{2}}\textcolor{white}{Homi Bhabha National Institute, Training School Complex, Anushakti Nagar, Mumbai 400085, India}}\\
    
    \vspace{0em} % Add vertical space
            \end{center}
            
            \begin{center}
                \begin{tikzpicture}[remember picture,overlay]
                     \node [anchor=north west, inner sep=0cm] at ([xshift=4cm,yshift=-4cm]current page.north west)
                    {\includegraphics[width=8cm,height=8cm]{imsclogo.png}}; % Adjust position and image
                    \node [anchor=north east, inner sep=0cm] at ([xshift=-2cm,yshift=-4cm]current page.north east)
                    {\includegraphics[width=8cm,height=8cm]{hbni.png}}; % Adjust position and image
                \end{tikzpicture}
            \end{center}
        \end{minipage}
    }
\end{center}

\vspace{-0.8cm}
% Abstract

\coloredsection{vibrantblue!60!white}{Abstract}{}
\coloredsubsection{vibrantblue!10!white}{
%\lipsum[1]
 We derive the full fourth-order TCL (TCL4) generator for the spin-boson model (SBM) for arbitrary odd spectral densities and show that the steady state matches the quantum mean-force Gibbs state up to $O(\lambda^2)$. Our results show that the commonly used TCL2 master equation often overestimates non-Markovianity. Benchmarking against analytical results for Ohmic spectral densities and the numerically exact Hierarchical Equations of Motion confirms the accuracy of TCL4. These results offer a general and computationally efficient method for studying steady-state and dynamical properties of a wide class of quantum systems, including semiconductor double quantum dots. 


%We investigate the steady-state and dynamical properties of the spin-boson model in the weak-coupling regime. Using higher-order quantum master equations, we analytically demonstrate that the second-order steady state matches with the quantum mean force Gibbs state, capturing environment-induced deviations from the textbook Gibbs state form. Extending to dynamics, we derive the complete fourth-order TCL generator and benchmark it against exact methods, revealing corrections to non-Markovianity and providing a computationally efficient tool for studying open quantum systems. Applications to semiconductor double-quantum-dot systems highlight the physical relevance of our results.

%\textbf{\Large Keywords:}\textbf{ Norbixin; Z-scan Technique; Third-order Optical Nonlinearity; Nonlinear Optics;}
%\vspace{-1em}
}

\begin{multicols}{2} % Two columns

\coloredsection{highlightgreen!60!white}{Background}{}
\coloredsubsection{highlightgreen!10!white}{
\textbf{TCL master equation: }\\
For a quantum state $\hat{\rho}(t)$, the time-convolutionless (TCL) ME has the following form
\begin{align}\label{eqn_TCL}
    \dot{\hat{\rho}}_S(t) &= \sum_{n=0}^\infty \lambda^{2n} \mathcal{F}^{(2n)}(t) [\hat{\rho}(t)].
\end{align}
Here, $\mathcal{F}^{(0)}(t) [\hat{\rho}(t)] = -i [\hat{H}_S, \hat{\rho}(t)]$ is the free Hamiltonian part, $\mathcal{F}^{(2)}(t)$ is the familiar BR-ME part and the higher order terms correspond to corrections to the BR-ME. The SS of the TCL-ME can then be calculated perturbatively by solving the following equation
\begin{align}\label{eqn_TCl_steady_state_perturbative}
\left( \mathcal{F}^{(0)} + \lambda^2 \mathcal{F}^{(2)} + ... \right) [\hat{\rho}^{(0)} + \lambda^2 \hat{\rho}^{(2)} + ...] = 0.
\end{align}
Here, $\hat{\rho}^{(0)}$ is the Gibbs state and the higher-order terms are corrections over it.
\vspace{1cm}\\
\textbf{Mean force Gibbs state: }\\
Mean force Gibbs state (MFGS) is defined as the 
 
\begin{equation}\label{eqn_mfgs_defn}
    \hat{\rho}_{eq} = \text{Tr}_E Z^{-1} e^{- \beta \hat{H}_{SE}},
 \end{equation} 
where $\hat{H}_{SE}$ is the full SE Hamiltonian, $Z$ is a normalization constant and $\beta$ is the inverse temperature of the environment.
}
% Objectives

% Methodology
\coloredsection{highlightgreen!60!white}{Spin-boson model}{}
\coloredsubsection{highlightgreen!10!white}{
% Methodology content here
\begin{minipage}[t]{\linewidth}
\begin{Box3}{}

\begin{itemize}
  \item[\circleicon{red}] The general Hamiltonian for the spin-boson model (SBM) is given by:
\begin{equation}
   \hat{H}_{SE} = \hat{H}_{S} + \hat{H}_{E} + \hat{H}_{I}
\end{equation}
where,
\begin{equation*}
    \hat{H}_{I} = \lambda \, \hat{A} \otimes \hat{B}.
\end{equation*}


\item[\circleicon{blue}] Spectral density is defined as 
\begin{equation}   
   J(\omega) = \sum_{k} \frac{c_{k}^{2}}{m_{k} \omega_{k}} \, \delta(\omega - \omega_{k}).
\end{equation}


\item[\circleicon{highlightgreen}] TCL-ME for SBM in Bloch vector form is written as
\begin{equation}
\dot{\mathbf{v}}(t) = \sum_{n=0}^{\infty} \lambda^{2n} \, F^{(2n)}(t) \, \mathbf{v}(t)
\end{equation}
where 
\[
F^{(2n)}_{mn}(t) \equiv \operatorname{Tr}\!\left\{ \hat{\sigma}_{m} \, F^{(2n)}(t)\!\left[ \hat{\sigma}_{n} \right] \right\}
\]
\end{itemize} 

\tcblower

\begin{tcolorbox}[colback=white!80!lightgray,colframe=white!0!red,width=\linewidth]
\[
\hat{H}_{S} \equiv \tfrac{\Omega}{2} \, \hat{\sigma}_{3},
\]
\[
\hat{H}_{E} = \sum_{k} \left( \frac{\hat{p}_{k}^{2}}{2 m_{k}} + \tfrac{1}{2} m_{k} \omega_{k}^{2} \hat{q}_{k}^{2} \right), 
\]
\[
\hat{A} = a_{3} \hat{\sigma}_{3} - a_{1} \hat{\sigma}_{1},
\]
\[
\hat{B} = \sum_{k} c_{k} \hat{q}_{k}. 
\]

\end{tcolorbox}

\begin{tcolorbox}[colback=white!80!accentorange,colframe=white!10!red,width=\linewidth]
$\lambda$ $\rightarrow$ system environment coupling parameter. \\
$A,B$ $\rightarrow$ system and environment operator. \\
$\hat{\sigma}_{i}$ $\rightarrow$  Pauli matrices.\\
\end{tcolorbox}

\begin{tcolorbox}[colback=white!80!vibrantblue,colframe=white!10!blue,width=\linewidth]
$m_{k}$, $\omega_{k}$ $\rightarrow$ mass, frequency of k^{\text{th}} oscillator.\\
$c_{k}$ $\rightarrow$ SE coupling. 

\end{tcolorbox}
\end{Box3}
\end{minipage}
}

\end{multicols}

\coloredsection{vibrantblue!50!white}{Results}{}

\begin{multicols}{2} % Two columns

%\coloredsection{highlightgreen!60!white}{Background}{}
\coloredsubsection{vibrantblue!5!white}{
\begin{center}
     \textbf{Application to the Double-Quantum-Dot System } 
\end{center}
\\
\begin{itemize}
    \item[\circleicon{blue}]  Dynamics of
a semiconductor Quantum-Double-Dot (DQD) system, which can be effectively modeled by the SBM. The SBM parameters are related to the DQD physical parameters (detuning $\epsilon$ and inter-dot tunneling $t_c$) as $a_{1} = \frac{2 t_{c}}{\Omega}$, $    a_{3} = \frac{\epsilon}{\Omega}$ and $\Omega^2 = \epsilon^2 + 4 t_c^2$.

\item[\circleicon{blue}] The DQD is coupled to a phononic bath, described by the spectral density
\end{itemize}
\begin{equation}
  J(\omega) = \gamma \omega \, \left[1-\text{sinc}\left(\frac{\omega}{\omega_c}\right) \right]\, \exp\left\{-\frac{\omega^2}{2\: \omega_{\text{max}}^2}\right\}, \label{eqn_J_DQD}
\end{equation}
where, we have $\text{sinc}(x) \equiv \sin(x)/x$. The parameter $\omega_{\text{max}}$ serves as the upper cut-off frequency, while $\omega_{c} = c_{s}/d$, where $c_{s}$ is the speed of sound in the substrate and $d$ is the inter-dot distance.
\vspace{0.5cm}
\begin{figure}[H]
    \resizebox{14cm}{10cm}{{\includegraphics{Sigma1Plot.pdf}}}
    \resizebox{14cm}{10cm}{{\includegraphics{Sigma2Plot.pdf}}}
     \resizebox{14cm}{10cm}{{\includegraphics{Sigma3Plot.pdf}}}
    \caption{Evolution of Pauli matrix expectation values (a) $\langle\hat{\sigma}_1(t)\rangle$, (b) $\langle\hat{\sigma}_2(t)\rangle$, and (c) $\langle\hat{\sigma}_3(t)\rangle$ for a DQD system using TCL2 (blue lines) and TCL4 (red lines) master equations. The model parameters are 
$\epsilon = 1$, $t_c = 0.5$, $\gamma\lambda^2 = 0.4$, $\beta = 1$, 
$\omega_{\text{max}} = 1$, and $\omega_c = 1$. 
The initial expectation values are 
$\langle \hat{\sigma}_1 \rangle = 0$, 
$\langle \hat{\sigma}_2 \rangle = 0$, and 
$\langle \hat{\sigma}_3 \rangle = -0.5$.}%
\end{figure}


\begin{center}
  \color{black}{\textbf{Numerical verification and benchmarking}}
\end{center}

\begin{itemize}
    \item[\circleicon{blue}] We compare TCL2 and TCL4 dynamics with HEOM for the Ohmic spectral density with Drude cutoff case, quantified by the fidelity between $\hat{\rho}_{\text{TCL}}$ and $\hat{\rho}_{\text{H}}$
    \begin{equation}
        F(\hat{\rho}_{\text{TCL}}(t), \hat{\rho}_{\text{H}}(t)) = \text{Tr} \sqrt{\sqrt{\hat{\rho}_{\text{H}}(t)} \hat{\rho}_{\text{TCL}}(t) \sqrt{\hat{\rho}_{\text{H}}(t)}}.
    \end{equation}
\end{itemize}

\begin{figure}[H]
    \resizebox{18cm}{15cm}{{\includegraphics{fidelity_tcl2_tcl4_Nk_32_Md_2.pdf}}}
    \resizebox{24cm}{14.5cm}{{\includegraphics{TCL4Verification.pdf}}}
    % \resizebox{14cm}{14cm}{{\includegraphics{AntiPredifftcl.pdf}}}
    \caption{These figures plot a) One minus the Fidelity of the system state
evolved by HEOM and TCL-ME (blue line is TCL2 and red line is TCL4), b)Relative difference between asymptotic TCL4 generator elements for the Ohmic–Drude case, obtained from our general odd-spectral-density result and from specialized calculations, plotted against the number of Matsubara terms..}%
\end{figure}
}

%\coloredsection{highlightgreen!60!white}{Background}{}
\coloredsubsection{vibrantblue!5!white}{
\begin{center}
    \textbf{Non-Markovian Effects with TCL4 }
\end{center}
\\
\begin{itemize}
    \item[\circleicon{blue}] We quantify the non-Markovianity of TCL2 and TCL4 dynamics for the SBM using the BLP\cite{breuer2009measure} measure for an Ohmic spectral density with Drude cutoff $J_D(\omega) = \frac{\gamma  \Lambda ^2 \omega }{\Lambda ^2+\omega ^2}$.
    \item[\circleicon{blue}] The degree of non-Markovianity of dynamics is defined by 
  \begin{equation}
    \mathcal{N}(\Phi) = \max_{\hat{\rho}_{1,2}(0)} \int_{\sigma > 0} dt\, \sigma(t, \hat{\rho}_{1,2}(0)),
   \label{eqn_blp_measure}
  \end{equation}
where $\sigma(t, \hat{\rho}_{1,2}(0)) = \frac{d}{dt} D[\Phi_t(\hat{\rho}_{1}(0)), \Phi_t(\hat{\rho}_{2}(0))]
$.
\end{itemize}

\begin{figure}[H]
    \resizebox{14cm}{11cm}{{\includegraphics{AntiPtcl2lamdaTplot.pdf}}}
    \resizebox{14cm}{11cm}{{\includegraphics{AntiPtcl4lamdaTplot.pdf}}}
     \resizebox{14cm}{11cm}{{\includegraphics{AntiPredifftcl.pdf}}}
    \caption{These figures plot the BLP measure ($N(\Phi)$) as logarithmic color plot calculated using (a) TCL2 and (b) TCL4 as a function of $\Lambda $ and $T$ Part (c) plots the difference between these 2 quantities ($N(\Phi_{\text{TCL4}})-N(\Phi_{\text{TCL2}})$). The black line marks the Markovian regime from the resonance condition. The BLP measure is maximized over $400$ antipodal Bloch sphere pairs.}%
\end{figure}
}

\coloredsection{accentorange!80!white}{Summary}{}
\coloredsubsection{accentorange!10!white}{
\begin{minipage}[t]{\linewidth}
\begin{itemize}
   \item[\circleicon{black}]  For a generic SBM, we proved that the steady state coincides with the MFGS up to $O(\lambda^{2})$ using the TCL4 generator, assuming only an odd spectral density
   $J(\omega)$.
  \item[\circleicon{black}] We report that for Ohmic-Drude spectral density, TCL2 overestimates non-Markovianity across a wide range of $T$ and $\Lambda$.
  \item[\circleicon{black}] We benchmarked the fourth-order TCL generator against Ohmic spectral density with Drude analytical results and the exact HEOM method.
\end{itemize}
\end{minipage}
}

\coloredsection{vibrantblue!60!white}{References}{}
\coloredsubsection{vibrantblue!10!white}{
\begin{minipage}[t]{\linewidth}
    
% % References content here
% \bibliographystyle{unsrt}
% \bibliography{name}


\begin{thebibliography}{9}
    \bibitem{prem2025equiv} Prem Kumar, K. P. Athulya, \& Sibasish Ghosh. (2025). 
    Equivalence between the second order steady state for the spin-boson model and its quantum mean force Gibbs state. 
    \textit{Phys. Rev. B}, \textbf{111}(11), 115423.

    \bibitem{prem2025tcl4} Prem Kumar, K. P. Athulya,\& Sibasish Ghosh. (2025). 
    Asymptotic TCL4 Generator for the Spin-Boson Model: Analytical Derivation and Benchmarking. 
    \textit{arXiv preprint arXiv:2506.17009}.

    \bibitem{breuer2009measure} Heinz-Peter Breuer, Elsi-Mari Laine, \& Jyrki Piilo. (2009). 
    Measure for the degree of non-Markovian behavior of quantum processes in open systems. 
    \textit{Phys. Rev. Lett.}, \textbf{103}(21), 210401.
\end{thebibliography}

%\begin{enumerate}
  %  \item Prem Kumar, K. P. Athulya, \& Sibasish Ghosh. (2025). Equivalence between the second order steady state for the spin-boson model and its quantum mean force Gibbs state. \textit{Phys. Rev. B}, \textbf{111}(11), 115423.

 %   \item Prem Kumar, K. P. Athulya,\& Sibsish Ghosh (2025). Asymptotic TCL4 Generator for the Spin-Boson Model: Analytical Derivation and Benchmarking. \textit{arXiv preprint arXiv:2506.17009}.

 %   \item Heinz-Peter Breuer, Elsi-Mari Laine, \& Jyrki Piilo. (2009). Measure for the degree of non-Markovian behavior of quantum processes in open systems. \textit{Phys. Rev. Lett.}, \textbf{103}(21), 210401.



%\end{enumerate}


\end{minipage}
}

\end{multicols}


\begin{tikzpicture}[remember picture, overlay]
    \fill[nottblue] (current page.south west) rectangle ([yshift=2cm]current page.south east);
\end{tikzpicture}
\end{document}

