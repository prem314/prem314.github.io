\documentclass{a0poster}
\usepackage[margin=0cm, paperwidth=90cm, paperheight=120cm]{geometry}
\usepackage{poster}
\usepackage{subcaption}

% Adjustable text sizing for the poster (edit these to resize everything)
\newcommand{\posterbodyfontsize}{28} % main text size
\newcommand{\posterbodybaselineskip}{34}
\newcommand{\posterheadingfontsize}{38} % mid-level headings
\newcommand{\posterheadingbaselineskip}{44}
\newcommand{\postersectionfontsize}{64} % top colored section titles (Abstract, Background, Results, etc.)
\newcommand{\postersectionbaselineskip}{66}
\newcommand{\posterbgsubfontsize}{30} % small headings inside Background column text
\newcommand{\posterbgsubbaselineskip}{34}
\newcommand{\posterqrcode}{img/qr_code_tcl4_2_poster.pdf} % full path for the QR image

\newcommand{\posterheading}[1]{{\fontsize{\posterheadingfontsize}{\posterheadingbaselineskip}\selectfont #1}}
\newcommand{\postersectiontitle}[1]{{\fontsize{\postersectionfontsize}{\postersectionbaselineskip}\selectfont\textbf{#1}}}
\newcommand{\bgtitle}[1]{{\fontsize{\posterbgsubfontsize}{\posterbgsubbaselineskip}\selectfont\textbf{#1}}}

% Override colored section box to use adjustable section title size
\renewcommand{\coloredsection}[3]{%
    \begin{center}%
        \begin{tcolorbox}[
            enhanced,
            colback=#1,
            colframe=#1!70,
            arc=6pt,
            boxrule=3pt,
            width=0.6\linewidth,
            ]
            \centering
            \postersectiontitle{#2}\\[3ex]
            #3 % Content added here
        \end{tcolorbox}%
    \end{center}%
}

\AtBeginDocument{\fontsize{\posterbodyfontsize}{\posterbodybaselineskip}\selectfont}

\begin{document}
\begin{center}
    \colorbox{nottblue!100}{%
        \begin{minipage}[t]{\textwidth} % Adjusted vertical alignment
            \vspace{0.8em} % Add vertical space
            \begin{center}
                {\fontsize{85pt}{85pt}\selectfont\textbf{\textcolor{white}{Asymptotic TCL4 Generator for the Spin-Boson Model: Analytical Derivation and Benchmarking}}}\\[1ex] % Increased font size
		    \Large \textit{\textcolor{white}{ \textbf{Prem Kumar} }}\textsuperscript{\textcolor{white}{1,2}}\textcolor{white}{, } \textit{\textcolor{white}{K. P. Athulya}}\textsuperscript{\textcolor{white}{1,2}}\textcolor{white}{, } \textit{\textcolor{white}{Sibasish Ghosh}}\textsuperscript{\textcolor{white}{1,2}}\\
    \textit{\textsuperscript{\textcolor{white}{1}}\textcolor{white}{ Optics \& Quantum Information Group, The Institute of Mathematical Sciences, CIT Campus, Taramani, Chennai 600113, India}}\\

    \textit{\textsuperscript{\textcolor{white}{2}}\textcolor{white}{Homi Bhabha National Institute, Training School Complex, Anushakti Nagar, Mumbai 400085, India}}\\
    
    \vspace{0em} % Add vertical space
            \end{center}
            
            \begin{center}
                \begin{tikzpicture}[remember picture,overlay]
                     \node [anchor=north west, inner sep=0cm] at ([xshift=4cm,yshift=-4cm]current page.north west)
                    {\includegraphics[width=8cm,height=8cm]{img/imsclogo.png}}; % Adjust position and image
                    \node [anchor=north east, inner sep=0cm] at ([xshift=-2cm,yshift=-4cm]current page.north east)
                    {\includegraphics[width=8cm,height=8cm]{img/hbni.png}}; % Adjust position and image
                \end{tikzpicture}
            \end{center}
        \end{minipage}
    }
\end{center}

% Space between title strip and first content row
\vspace{1cm}
% Abstract + Background + QR code strip
\noindent
\begin{minipage}[t]{0.28\textwidth}
  \coloredsection{accentorange!70!white}{Abstract}{}
  \coloredsubsection{accentorange!10!white}{~}{
%\vspace{0.8em}% uniform top padding inside Abstract text
We derive the full fourth-order corrections to the open quantum system dynamics (in system-environment coupling strength), for the spin-boson model (SBM) for arbitrary odd physical spectral densities. We apply this result to solve for the dynamics of a semiconductor double quantum dot, which can be modeled as a SBM under certain conditions. Our results show that the commonly used second order master equation often overestimates the non-Markovianity at large temperatures. We also benchmark our results against analytical results for Ohmic spectral densities and the numerically exact Hierarchical Equations of Motion. These results offer a general and computationally efficient method for studying higher order corrections to the dynamics of a wide class of open quantum systems.
}
\end{minipage}
\hfill
\begin{minipage}[t]{0.52\textwidth}
  \coloredsection{highlightgreen!60!white}{Background}{}
  \coloredsubsection{highlightgreen!10!white}{~}{
  \begingroup
  \setlength{\columnsep}{1.2cm}
  \setlength{\leftskip}{0.8cm}
  \setlength{\rightskip}{2cm}
  \begin{multicols}{2}
\textbf{Spin-Boson model:}
  The Hamiltonian is
  \begin{equation*}
     \hat{H}_{SE} = \hat{H}_{S} + \hat{H}_{E} + \hat{H}_{I}, \qquad 
     \hat{H}_{I} = \lambda \, \hat{A} \otimes \hat{B}.
  \end{equation*}

  

  Its ingredients are
  \begin{align*}
      \hat{H}_{S} &= \tfrac{\Omega}{2}\, \hat{\sigma}_{3},\\
      \hat{H}_{E} &= \sum_{k} \Big( \tfrac{\hat{p}_{k}^{2}}{2 m_{k}} + \tfrac{1}{2} m_{k} \omega_{k}^{2} \hat{q}_{k}^{2} \Big).
  \end{align*}
Here, $\hat{A} = a_{3} \hat{\sigma}_{3} - a_{1} \hat{\sigma}_{1}$, $\hat{B} = \sum_{k} c_{k} \hat{q}_{k}$, $\hat{\sigma}_i$ are Pauli matrices; $m_k,\omega_k,c_k$ are the mass, frequency, and coupling of the $k^{\text{th}}$ mode.
Spectral density: $J(\omega) = \sum_{k} \frac{c_{k}^{2}}{m_{k} \omega_{k}} \, \delta(\omega - \omega_{k}).$

\columnbreak



	  \textbf{TCL master equation:}
  For a quantum state $\hat{\rho}(t)$, the time-convolutionless (TCL) master equation is
  \begin{align*}
      \dot{\hat{\rho}}_S(t) &= \sum_{n=0}^\infty \lambda^{2n} \mathcal{F}^{(2n)}(t)\big[\hat{\rho}(t)\big].
  \end{align*}
Here $\lambda$ is the system–environment coupling parameter. The zeroth-order part $\mathcal{F}^{(0)}(t)[\hat{\rho}(t)] = -i [\hat{H}_S, \hat{\rho}(t)]$ gives free evolution, $\mathcal{F}^{(2)}(t)$ is the Bloch–Redfield contribution, and higher even orders add systematic corrections. 












  \end{multicols}
  \endgroup
  }
\end{minipage}
\hfill
\begin{minipage}[t]{0.18\textwidth}
\coloredsection{accentorange!70!white}{Scan}{}
\coloredsubsection{accentorange!10!white}{~}{
    \begin{center}
      \includegraphics[width=0.9\linewidth]{\posterqrcode}\\[0.6em]
      \posterheading{\textbf{Scan for PDF of this poster and paper}}
    \end{center}
  }
\end{minipage}

% Space between top row and Results
\vspace{1cm}

\coloredsection{vibrantblue!50!white}{Results}{}

\begin{multicols}{2} % Two columns

%\coloredsection{highlightgreen!60!white}{Background}{}
\coloredsubsection{vibrantblue!5!white}{~}{
\begin{center}
     \posterheading{\textbf{Application to the Double-Quantum-Dot System}} 
\end{center}
\vspace{0.5cm}
\begin{itemize}
    \item[\circleicon{blue}]  Dynamics of
a semiconductor Quantum-Double-Dot (DQD) system, which can be effectively modeled by the SBM. The SBM parameters are related to the DQD physical parameters (detuning $\epsilon$ and inter-dot tunneling $t_c$) as $a_{1} = \frac{2 t_{c}}{\Omega}$, $    a_{3} = \frac{\epsilon}{\Omega}$ and $\Omega^2 = \epsilon^2 + 4 t_c^2$.

\item[\circleicon{blue}] The DQD is coupled to a phononic bath, described by the spectral density
\end{itemize}
\begin{equation*}
  J(\omega) = \gamma \omega \, \left[1-\text{sinc}\left(\frac{\omega}{\omega_c}\right) \right]\, \exp\left\{-\frac{\omega^2}{2\: \omega_{\text{max}}^2}\right\},
\end{equation*}
where, we have $\text{sinc}(x) \equiv \sin(x)/x$. The parameter $\omega_{\text{max}}$ serves as the upper cut-off frequency, while $\omega_{c} = c_{s}/d$, where $c_{s}$ is the speed of sound in the substrate and $d$ is the inter-dot distance.
\vspace{0.5cm}
\begin{figure}[H]
    \resizebox{14cm}{10cm}{{\includegraphics{img/Sigma1Plot.pdf}}}
    \resizebox{14cm}{10cm}{{\includegraphics{img/Sigma2Plot.pdf}}}
     \resizebox{14cm}{10cm}{{\includegraphics{img/Sigma3Plot.pdf}}}
    \caption{Evolution of Pauli matrix expectation values (a) $\langle\hat{\sigma}_1(t)\rangle$, (b) $\langle\hat{\sigma}_2(t)\rangle$, and (c) $\langle\hat{\sigma}_3(t)\rangle$ for a DQD system using TCL2 (blue dotted lines) and TCL4 (red lines) master equations. The model parameters are 
$\epsilon = 1$, $t_c = 0.5$, $\gamma\lambda^2 = 0.4$, $\beta = 1$, 
$\omega_{\text{max}} = 1$, and $\omega_c = 1$. 
The initial expectation values are 
$\langle \hat{\sigma}_1 \rangle = 0$, 
$\langle \hat{\sigma}_2 \rangle = 0$, and 
$\langle \hat{\sigma}_3 \rangle = -0.5$.}%
\end{figure}


\begin{center}
  \color{black}{\posterheading{\textbf{Numerical verification and benchmarking}}}
\end{center}

\begin{itemize}
	\item[\circleicon{blue}] We compare TCL2 and TCL4 dynamics with HEOM for the Ohmic spectral density with Drude cutoff case $\left(J_D(\omega) = \frac{\gamma  \Lambda ^2 \omega }{\Lambda ^2+\omega ^2}\right)$, quantified by the fidelity between $\hat{\rho}_{\text{TCL}}$ and $\hat{\rho}_{\text{H}}$
    \begin{equation}
        F(\hat{\rho}_{\text{TCL}}(t), \hat{\rho}_{\text{H}}(t)) = \text{Tr} \sqrt{\sqrt{\hat{\rho}_{\text{H}}(t)} \hat{\rho}_{\text{TCL}}(t) \sqrt{\hat{\rho}_{\text{H}}(t)}}.
    \end{equation}
\end{itemize}

\begin{figure}[H]
    \resizebox{18cm}{15cm}{{\includegraphics{img/fidelity_tcl2_tcl4_Nk_32_Md_2_tau_5.0_Lambda_1_T_1.0.pdf}}}
    \resizebox{24cm}{14.5cm}{{\includegraphics{img/TCL4Verification.pdf}}}
    % \resizebox{14cm}{14cm}{{\includegraphics{img/AntiPredifftcl.pdf}}}
    \caption{These figures plot a) One minus the Fidelity of the system state
evolved by HEOM and TCL-ME (blue dotted line is TCL2 and red line is TCL4), b) Relative difference between asymptotic TCL4 generator elements for the Ohmic–Drude case, obtained from our general odd-spectral-density result and from specialized calculations, plotted against the number of Matsubara terms used in the latter calculation.}%
\end{figure}
}

%\coloredsection{highlightgreen!60!white}{Background}{}
\coloredsubsection{vibrantblue!5!white}{~}{
\begin{center}
    \posterheading{\textbf{Non-Markovian Effects with TCL4}}
\end{center}
\vspace{0.5cm}
\begin{itemize}
    \item[\circleicon{blue}] We quantify the non-Markovianity of TCL2 and TCL4 dynamics for the SBM using the BLP measure (Breuer, Laine \& Piilo, 2009) for an Ohmic spectral density with Drude cutoff.
    \item[\circleicon{blue}] The degree of non-Markovianity of dynamics is defined by 
  \begin{equation*}
    N(\Phi) = \max_{\hat{\rho}_{1,2}(0)} \int_{\sigma > 0} dt\, \sigma(t, \hat{\rho}_{1,2}(0)),
  \end{equation*}
where $\sigma(t, \hat{\rho}_{1,2}(0)) = \frac{d}{dt} D[\Phi_t(\hat{\rho}_{1}(0)), \Phi_t(\hat{\rho}_{2}(0))]
		$. Here, $\Phi_t$ is the dynamical map and $D[\rho, \sigma]$ is trace distance between states $\rho$ and $\sigma$.
\end{itemize}

\begin{figure}[H]
    \centering
    \begin{subfigure}[b]{0.48\linewidth}
        \includegraphics[width=\linewidth]{img/AntiPtcl2lamdaTplot.pdf}
        \caption{BLP measure using asymptotic TCL2}
    \end{subfigure}
    \hfill
    \begin{subfigure}[b]{0.48\linewidth}
        \includegraphics[width=\linewidth]{img/AntiPtcl4lamdaTplot.pdf}
        \caption{BLP measure using asymptotic TCL4}
    \end{subfigure}

    \vspace{0.5cm}

    \begin{subfigure}[b]{0.48\linewidth}
        \includegraphics[width=\linewidth]{img/AntiPredifftcl.pdf}
        \caption{Difference: $N(\Phi_{\text{TCL4}})-N(\Phi_{\text{TCL2}})$}
    \end{subfigure}
    \hfill
    \begin{subfigure}[b]{0.48\linewidth}
        \includegraphics[width=\linewidth]{img/TCLNormPlot.pdf}
        \caption{Norm Ratio: $||F^{(4)}||_2/||F^{(2)}||_2$}
    \end{subfigure}
    \caption{These figures plot the BLP measure ($N(\Phi)$) as logarithmic color plot calculated using (a) TCL2 and (b) TCL4 as a function of $\Lambda$ and $T$. Part (c) plots the difference between these two quantities ($N(\Phi_{\text{TCL4}})-N(\Phi_{\text{TCL2}})$), while (d) shows the $L_2$ norm ratio $||F^{(4)}||_2/||F^{(2)}||_2$, highlighting the regime of validity of the perturbative method. The black line marks the Markovian regime from the resonance condition. The BLP measure is maximized over $400$ antipodal Bloch sphere pairs.}%
\end{figure}
}

\end{multicols}

% Two-column footer: References (left) and Summary (right)
\noindent
\begin{minipage}[t]{0.48\textwidth}
  \coloredsection{highlightgreen!60!white}{References}{}
  \coloredsubsection{highlightgreen!10!white}{~}{
  \begin{itemize}[leftmargin=0.8em, label={}, itemsep=0.4em, topsep=0pt, parsep=0pt]
    \item Prem Kumar, K. P. Athulya, \& Sibasish Ghosh. (2025). Equivalence between the second order steady state for the spin-boson model and its quantum mean force Gibbs state. \textit{Phys. Rev. B}, \textbf{111}(11), 115423.
    \item Prem Kumar, K. P. Athulya, \& Sibasish Ghosh. (2025). Asymptotic TCL4 Generator for the Spin-Boson Model: Analytical Derivation and Benchmarking. \textit{arXiv preprint arXiv:2506.17009}.
    \item Heinz-Peter Breuer, Elsi-Mari Laine, \& Jyrki Piilo. (2009). Measure for the degree of non-Markovian behavior of quantum processes in open systems. \textit{Phys. Rev. Lett.}, \textbf{103}(21), 210401.
  \end{itemize}
  }
\end{minipage}
\hfill
\begin{minipage}[t]{0.48\textwidth}
\coloredsection{accentorange!70!white}{Summary}{}
  \coloredsubsection{accentorange!10!white}{~}{
  \begin{itemize}
	\item[\circleicon{black}] We analytically derive the full TCL4 generator in large time limit for a generic SBM for arbitrary odd physical spectral densities.
    \item[\circleicon{black}] For Ohmic-Drude spectral density, TCL2 overestimates non-Markovianity at large values of temperature.
    \item[\circleicon{black}] TCL4 is benchmarked against analytical Drude results and exact HEOM, showing closer agreement across tested regimes.
  \end{itemize}
  }
\end{minipage}



\begin{tikzpicture}[remember picture, overlay]
    \fill[nottblue] (current page.south west) rectangle ([yshift=2cm]current page.south east);
\end{tikzpicture}
\end{document}
