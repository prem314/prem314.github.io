\documentclass[a0paper,fleqn]{betterportraitposter}


% PACKAGES
\usepackage[none]{hyphenat}
\usepackage{hyperref}
\hypersetup{
  colorlinks=true,
  linkcolor=black, % color of internal links (sections, pages, etc.)
  citecolor=black, % color of citations
  urlcolor=black   % color of URLs
}
\renewcommand{\equationautorefname}{Eq.}




\usepackage{xcolor}
\usepackage{amsmath}
\usepackage{lipsum}
\usepackage{comment}
\usepackage{braket}
% Redefine math environments to display all math in red


%%%% Uncomment the following commands to customise the format

%% Setting the height of the top and bottom (colored) bars
%% Uncomment either of the following lines it you want to change the defaults heights of the top or bottom bars.
%variable that holds the length of the lower bar
\newcommand{\lowerBarLength}{0.1}

\newcommand{\upperBarLength}{0.16}



\setlength{\mainfindingheight}{\upperBarLength \paperheight} % Top bar
\setlength{\bottomboxheight}{\lowerBarLength \paperheight} % Bottom bar

%% Setting the page margin

%% Changing font sizes

% Text font
\renewcommand{\fontsizestandard}{\fontsize{25}{30} \selectfont}

% Main column font
%\renewcommand{\fontsizemain}{\fontsize{112}{140} \selectfont}
\renewcommand{\fontsizemain}{\fontsize{100}{120} \selectfont}

% Title font
%\renewcommand{\fontsizetitle}{\fontsize{64}{77} \selectfont}
\renewcommand{\fontsizetitle}{\fontsize{53}{64} \selectfont}

% Author font
\renewcommand{\fontsizeauthor}{\fontsize{28}{35} \selectfont}

% Institution font
\renewcommand{\fontsizeauthor}{\fontsize{28}{35} \selectfont}

% Section font
\renewcommand{\fontsizesection}{\fontsize{40}{48} \selectfont}

%% Changing font sizes for a specific text segment
% Place the text inside brackets:
% {\fontsize{28}{35} \selectfont Your text goes here}

%% Changing colours
% Background of main claim box (options include: imperialblue, empirical, theory, methods and intervention
% Default is empirical
% \renewcommand{\maincolumnbackgroundcolor}{intervention}

% Font on main and bottom boxes
% \renewcommand{\maincolumnfontcolor}{empirical}

% You can add a custom RGB color like so (example here is University of Glagow blue):

%change this for color! {1, 35, 57} for blue.
%\definecolor{MyBlue}{RGB}{100, 35, 57}

\definecolor{MyBlue}{RGB}{0,62,116}

\definecolor{MyBlueDarker}{RGB}{0,39,80}

\renewcommand{\maincolumnbackgroundcolor}{MyBlue}
\renewcommand{\institutefontcolor}{gray}
\renewcommand{\authorfontcolor}{gray}




\begin{document}	

\begin{comment}

Things to add
\begin{enumerate}
    \item Remove text from this document.
    \item Start adjusting stuff around, optimizing space, if at all, only when you have removed all the rubbish from the poster.
    \item Do not overdo things. Avoid OCD. Better things to do. Just get done with this.
    \item Add the other plot.
    \item make the poster less verbose. Remove words like 'as shown in figure'. Get to the point.
    \item add colors to the figures.
\end{enumerate}
    
\end{comment}

%% Top box with main message
%\mainfinding{The form of quantum equilibrium state shows universality at strong coupling for anharmonic environment.}

%\mainfinding{Universality in the form of strong coupling quantum equilibrium state as we go to anharmonic environment.}

%\mainfinding{Universality of Strongly Coupled Quantum Equilibrium States for Anharmonic environment}

%\mainfinding{Result: Universality of Strongly Coupled Quantum Equilibrium States upon Anharmonic generalization of the environment}

%\mainfinding{Result: Equilibrium state of a quantum system does not depend upon the details of anharmonic environment it is strongly coupled to.}

\mainfinding{Main finding: The \textbf{equilibrium state} of a \textbf{quantum system}, \textbf{strongly coupled} with an \textbf{anharmonic environment}, shows \textbf{universal traits} independent of the details of the environment. }

%\mainfinding{Main finding: A quantum system in equilibrium shows universal traits when it is strongly coupled to an anharmonic environment, independent of the details of the environment.}

%\mainfinding{Main result: \textbf{Strong coupling quantum equilibrium state} shows \textbf{robustness} under \textbf{anharmonic generalization} of \textbf{environment}.}

%% Title, author and affiliations section
\titlebox{
    \textcolor{MyBlueDarker}{\title{Ultrastrong coupling limit to quantum mean force Gibbs state for anharmonic environment}}
    %\author{Prem Kumar and Sibasish Ghosh}
    
    %\institution{\fontsize{28}{35} \selectfont The Institute of Mathematical Sciences, Chennai}
    
    %\institution{\fontsize{28}{35} \selectfont Homi Bhabha National Institute, Mumbai, India}
    }
% End of title stuff
    




%% Central box with traditional content
\centerbox{ \begin{multicols}{3}

\section{\textcolor{MyBlue}{Motivation}}
\begin{enumerate}
    \item \textbf{\textcolor{MyBlue}{Generalized equilibrium state}} of a quantum system deviates from the textbook Gibbs state at non-negligible \textcolor{MyBlue}{\textbf{system-environment (SE) coupling}} \cite{trushechkin2022open}.
    \item The analytical expression for this state for harmonic environment model with large SE coupling is known \cite{ankerhold2003phase, cresser2021weak}.
    \item However, the harmonic environment assumption is not always accurate \cite{bramberger2020dephasing}. The present work generalizes this result to more general SE models \cite{kumar2024Ultrastrong}.
\end{enumerate}





\section{\textcolor{MyBlue}{Results}}
\begin{enumerate}
    \item The expression for the generalized equilibrium state remains unchanged even after significant \textcolor{MyBlue}{\textbf{anharmonic generalization}} of the environment.
    \item For an even larger class of SE models, although the state deviates from the harmonic result, the basis in which it diagonalizes remains unchanged.
\end{enumerate}






\section{\textcolor{MyBlue}{The harmonic environment model}}
\begin{center}
    \includegraphics[width=0.6\columnwidth]{img/cl_model.pdf}
\end{center}
This model involves a quantum system coupled to many uncoupled harmonic oscillators.
The full SE Hamiltonian is given as $\hat{H}_{\mathrm{SE}} = \hat{H}_S + \hat{H}_{\bar{S}}$, where $\hat{H}_S$ is the free system Hamiltonian and
\begin{equation}
	\hat{H}_{\bar{S}} = \sum_k\left[\frac{\hat{p}_k^2}{2 m_k} + \frac{1}{2} m_k \omega_k^2\left(\hat{q}_k- \alpha_k \hat{A}\right)^2 \right]. \label{eqn_CL_hamiltonian}
\end{equation}
Here, $\hat{p}_k$ and $\hat{q}_k$ are the momentum and position operator of the $k$th environment particle, and $\hat{A}$ is a system operator.





\section{\textcolor{MyBlue}{Generalized equilibrium state}}

Generalized equilibrium state of a system, given the SE Hamiltonian $\hat{H}_{\text{SE}}$ at inverse temperature $\beta$, is formally defined as:
\begin{equation}
	\hat{\rho}_{eq} = Z^{-1}\text{Tr}_{E}\left[ e^{-\beta \hat{H}_{\text{SE}}}\right].
\end{equation}
The following image plots the equilibrium state, $\rho(x,y)$, for a harmonic oscillator in position basis (which is known exactly for arbitrary SE coupling), as we increase the coupling strength (from left to the right image).
\begin{center}
\includegraphics[width= \columnwidth]{img/ho_mfgs.pdf} 
\end{center}


\columnbreak





\section{\textcolor{MyBlue}{Strong coupling equilibrium state}}

The \textcolor{MyBlue}{\textbf{strong coupling (SC) limit}} is defined as $\lim \alpha_k \to \infty$ and the corresponding equilibrium state has been recently determined to be \cite{cresser2021weak}
\begin{equation}
\hat{P}_i\hat{\rho}_{\text{SC}} \hat{P}_j = Z^{-1} \delta(i,j)  \exp \left\{ -\beta \hat{P}_i \hat{H}_S \hat{P}_i\right\} \label{eqn_GCL2_MFGS},
\end{equation}
where $\hat{P}_i$ is the projection operator on the $i$th degenerate subspaces of $\hat{A}$ ($\hat{P}_i$ being 1D projectors in the case of no degeneracy).





\section{\textcolor{MyBlue}{Why anharmonic environment model?}}

The harmonic environment model is known to fail for many physical setups, for example, in the case of \textcolor{MyBlue}{\textbf{electron transfer}} happening in the presence of 
\begin{enumerate}
    \item strongly coupled \textcolor{MyBlue}{\textbf{low-frequency intramolecular modes}}, or,
    \item an environment consisting of \textcolor{MyBlue}{\textbf{nonpolar liquids}} (see, for example, \cite{bramberger2020dephasing}).
\end{enumerate}





\section{\textcolor{MyBlue}{Generalization: GCL2 Model}}

If we generalize \autoref{eqn_CL_hamiltonian} as following \cite{Wang_2019}
\begin{equation}
\hat{H}_{\bar{S}} = \sum_k\left[\frac{\hat{p}_k^2}{2 m_k} + U_k(\hat{q}_k- \alpha_k \hat{A}) \right], \label{eqn_GCL2}
\end{equation}
then, under some physical assumption on the interaction potential $U_k(x)$, we prove that the SC equilibrium state (\autoref{eqn_GCL2_MFGS}) \textcolor{MyBlue}{\textbf{remains unchanged}}!
\begin{center}
\includegraphics[width =  \columnwidth]{img/trace_distance_vs_coupling_strength.pdf}
\end{center}
This figure plots, for a specific qutrit system, the trace distance between the SC equilibrium state proposed here and the numerically evaluated equilibrium state as a function of the coupling strength, for $U_k(x) = \sum_{n=1}^3 a_{2n} x^{2n}$ in \autoref{eqn_GCL2}. We note that the trace distance vanishes in the large coupling limit, irrespective of the form of $U_k(x)$.





\textcolor{MyBlue}{\textbf{Intuitive Proof:}} In the path integral approach, \autoref{eqn_GCL2} gives rise to the following action
\begin{equation}
S_k = \int_0^\beta dt \bigg\{ \frac{m_k}{2} \dot{q}_k(t)^2 + U_k(q_k(t) - \alpha_k A(t))\bigg\}. \label{eqn:CL environment action}
\end{equation}
In the SC limit ($\alpha_k \to \infty$), small variations in $A(t)$ cause large changes in the argument of $U_k(x)$, giving rise to large potential energy cost to that path, unless we have $q_k(t) \approx \alpha_k A(t)$. But, in the latter case, the kinetic energy cost ($m_k\dot{q}_k(t)^2/2$) of that path blows up instead. Either way, in SC limit, $A(t)= \text{constant}$ is intuitively imposed, which leads to the SC equilibrium state (\autoref{eqn_GCL2_MFGS}). See the paper for the detailed proof \cite{kumar2024Ultrastrong}.




\section{\textcolor{MyBlue}{Further Generalizations: GCL Model}}

If we further generalize \autoref{eqn_CL_hamiltonian} as following,
\begin{equation}
\hat{H}_{\bar{S}} = \sum_k\left[\frac{\hat{p}_k^2}{2 m_k} + V_k(\hat{q}_k, \alpha_k \hat{A}) \right],
\end{equation}
then, under some physically motivated assumptions on the form of $V_k(x,y)$, we find that although the form of the equilibrium state does change to


\columnbreak


\begin{equation}
\hat{P}_i\hat{\rho}_{\text{SC}} \hat{P}_j = Z^{-1} \delta(i,j) \exp \left\{ -\beta \hat{P}_i \hat{H}_S \hat{P}_i\right\} \text{Tr} e^{-\beta \hat{H}_I(A_i)},
\end{equation}
it still \textcolor{MyBlue}{\textbf{diagonalizes in the basis of $\hat{A}$}}. Here, the only difference being the extra $A_i$ dependent factor $\text{Tr} e^{-\beta \hat{H}_I(A_i)}$, where $A_i$ is the eigenvalue of $\hat{A}$ corresponding to $\hat{P}_i$.








\section{\textcolor{MyBlue}{Zwanzig Model}}

For a different class of SE model with $\hat{H}_{\bar{S}}$ given as \cite{Banerjee_2002}
\begin{equation}
\hat{H}_{\bar{S}} = \sum_{k=1}^N\left[ \frac{\hat{p}_k^2}{2 m_k} + U_k^{\text{free}}(\hat{q}_k) + \frac{\alpha_k}{2} U_k(\hat{q}_k - \hat{A}) \right] \label{eqn_new_spring_model},
\end{equation}
under the assumption that $U_k(x)$ has a unique global minima at $x=0$, the equilibrium state is again found to \textcolor{MyBlue}{\textbf{diagonalize in the basis of $\hat{A}$}}. 

\textcolor{MyBlue}{\textbf{For a discrete system}}, the equilibrium state has the same form as the harmonic environment case (\autoref{eqn_GCL2_MFGS}), albeit for a renormalized system Hamiltonian.
\begin{center}
\includegraphics[width = \columnwidth]{img/zwanzig.pdf}
\end{center}
This figure plots, for a specific qutrit system, the trace distance between the SC equilibrium state proposed here and the numerically evaluated equilibrium state as a function of the coupling strength, for the model described by \autoref{eqn_new_spring_model}.

\textcolor{MyBlue}{\textbf{For a continous variable system}} with the system Hamiltonian given as
$\hat{H}_S = \frac{\hat{p}^2}{2m} + V(\hat{q})$ and $\hat{A} \equiv \hat{q}$,
the equilibrium state shows deviation from the harmonic environment case, and is given as,
\begin{align}
\braket{ q | \rho_{\text{SC}} | q'} &= Z^{-1} \delta(q,q') \braket{ q |e^{-\beta \hat{H}_{\text{eff}}} | q'} \label{eqn:zwanzig SC result},\\
\hat{H}_{\text{eff}} &= \frac{\hat{p}^2}{2 M_{\text{eff}}} + V_{\text{eff}}(\hat{q}),
\end{align}
where, $M_{\text{eff}} = m + \sum_k m_k$ and $V_{\text{eff}}(x) = V(x) + \sum_{k=1}^N U_k^{\text{free}}(x)$. We note that this converges to the harmonic environment result (\autoref{eqn_GCL2_MFGS}) in the limit $M_{\text{eff}} \to \infty$.





\textcolor{MyBlue}{\textbf{Intuitive Proof:}} The action corresponding to \autoref{eqn_new_spring_model} is given as
\begin{equation}
S_k \equiv \int_0^\beta dt \bigg\{ \frac{m_k}{2} \dot{q}_k(t)^2 + U_k^{\text{free}}(q_k(t)) + \frac{\alpha_k}{2} U_k(q_k(t) - A(t))\bigg\}.
\end{equation}
In the SC limit ($\alpha_k \to \infty$), for paths with non-negligible contribution, the condition $q_k(t) = A(t)$ and $A(0) = A(\beta)$ gets imposed. The action simplifies as
\begin{equation}
S_k[A(t)] = \int_0^\beta dt \bigg\{ \frac{m_k}{2} \dot{A}(t)^2 + U_k^{\text{free}}(A(t))\bigg\}.
\end{equation}
Hence eliminating the path integral over the environment degrees of freedom and giving rise to the expression for the SC equilibrium state derived here.
{
\fontsize{18}{22}\selectfont
\begin{thebibliography}{9}

\bibitem{trushechkin2022open}
A.S.~Trushechkin, M.~Merkli, J.D.~Cresser, and J.~Anders, \textit{AVS Quantum Science}, vol.~4, no.~1, 2022.

\bibitem{ankerhold2003phase}
J.~Ankerhold, ``Phase space dynamics of overdamped quantum systems,'' \textit{Europhysics Letters}, vol.~61, 2003.

\bibitem{cresser2021weak}
J.D.~Cresser and J.~Anders, \textit{Physical Review Letters}, vol.~127, no.~25, p.~250601, 2021.

\bibitem{bramberger2020dephasing}
M.~Bramberger, I.~De Vega, \textit{Phys. Rev. A}, vol.~101, no.~1, 2020.

\bibitem{kumar2024Ultrastrong}
P.~Kumar and S.~Ghosh, \textit{arXiv preprint:2405.03044}, 2024.

\bibitem{Wang_2019}
F.~Wang and N.~Makri, \textit{J. Chem. Phys.}, vol.~150, no.~18, May 2019. \href{http://dx.doi.org/10.1063/1.5091725}{DOI: 10.1063/1.5091725}

\bibitem{Banerjee_2002}
D.~Banerjee, B.C.~Bag, S.K.~Banik, D.S.~Ray, \textit{Phys. Rev. E}, 2002. \href{http://dx.doi.org/10.1103/PhysRevE.65.021109}{DOI: 10.1103/PhysRevE.65.021109}

\end{thebibliography}
}

\end{multicols}

}

% This curly braket marks the end of central box



% Bottom box with QR code
\bottombox{
    %% QR code
    %\qrcode{img/qrcode_beyond_cl.pdf}{img/smartphoneWhite}{Scan QR code to get the full paper}
    \qrcode{img/qrcode_beyond_cl.pdf}{img/smartphoneWhite}{\textbf{Prem Kumar} and Sibasish Ghosh
    
    IMSC, Chennai
    
    HBNI, Mumbai, India
    
    premkr@imsc.res.in}
    % Comment out the line below out to hide logo
    \hfill\bottomboxlogo[height =  \lowerBarLength \paperheight]{img/logo.png}
    %\hfill\bottomboxlogo[width= 0.5 \textwidth]{img/hbni.png}
    % \hfill shifts the logo across so it meets the right hand side margin
    % Note that \bottomboxlogo takes an optional width argument. It defaults to the following: 
    % \hfill\bottomboxlogo[width=\textwidth]{<path_to_image_file>} 
    % where \textwidth is actually the width of a minipage which is defined in the \bottombox command of
    % betterportaitposter.cls It's a standard \includegraphics command in there, so easy to change if 
    % you need to add a border etc.
    
}
% End of bottom box


%\qrcode{img/qrcode}{img/smartphoneWhite}{
%\textbf{Take a picture} to
%\\download the full paper
%}


%% Compact QR code (comment the previous command and uncomment this one to switch)
%\compactqrcode{img/qrcode}{
%\textbf{Take a picture} to
%\\download the full paper
%}

%}

\end{document}
