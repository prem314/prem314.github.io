\documentclass[a4paper,11pt]{article}
\usepackage[left=1in, right=1in, top=1in, bottom=1in]{geometry}
\usepackage{enumitem}
\usepackage{xcolor}
\usepackage{comment} % enable \begin{comment} ... \end{comment} blocks
\usepackage[backend=biber,style=numeric,sorting=none]{biblatex}
\addbibresource{citation.bib}
\usepackage{hyperref}

\hypersetup{
    colorlinks=true,
    linkcolor=blue,
    filecolor=magenta,      
    urlcolor=blue,
}

\urlstyle{same}

\newcommand{\sectionheading}[1]{\subsection*{\textcolor{purple}{#1}}}

\begin{document}

\begin{center}
    \LARGE{\href{https://scholar.google.com/citations?hl=en&user=f-ZPkk0AAAAJ}{\textbf{Prem Kumar}}} \\[0.2cm]
    \large{Ph.D. Student} \\[0.1cm]
    \large{The Institute of Mathematical Sciences (IMSc), Chennai, India} \\[0.1cm]
    Website: \href{https://prem314.github.io}{prem314.github.io}\\
    Email: \href{mailto:premkr@imsc.res.in}{premkr@imsc.res.in}, \href{mailto:prem3141592@gmail.com}{prem3141592@gmail.com} \\[0.1cm]
\end{center}


\sectionheading{Education and Training}
\begin{itemize}[leftmargin=*,label={}]
    \item \textbf{Ph.D. in Theoretical Physics} \hfill 2021-Present \\
          Institute of Mathematical Sciences (IMSc), Chennai, India \\
          PhD supervisor: \href{https://scholar.google.com/citations?user=zr0J0lMAAAAJ&hl=en&oi=ao}{Prof. Sibasish Ghosh}
    \item \textbf{M.Sc. in Physics} \hfill 2019-2021 \\
          Institute of Mathematical Sciences (IMSc), Chennai, India \\
%          Project: Dynamics of open quantum Systems: Different approaches to non-Markovianity \\
          MSc project supervisor: \href{https://scholar.google.com/citations?user=zr0J0lMAAAAJ&hl=en&oi=ao}{Prof. Sibasish Ghosh}
          %MSc. thesis title: \href{https://drive.google.com/file/d/1WXMqmcygDxn_AjsxEjA2ZhBT_xq59dKc/view}{``Dynamics of open quantum systems: Different approaches to non-Markovianity''}.\\
    \item \textbf{B.Sc. in Physics, Maths, Electronics} \hfill 2016 - 2019\\
    Christ (Deemed to be University), Bangalore, India
\end{itemize}

\sectionheading{Research Highlights}
\begin{itemize}[leftmargin=*]
    \item \textbf{Corrections to quantum dissipative dynamics at higher order in the noise strength:}
    \begin{itemize}
        \item Analytical derivation of the 4th-order Time-Convolutionless (TCL4) generator for Spin-Boson models for a very large class of model parameters \cite{Kumar2025MF}.
        \item Analytical proof that the higher order corrections to steady state match exactly with the corresponding generalized Gibbs state \cite{Kumar2025MF}. 
        %A direct analytical proof of this equivalence in this generality has not been reported in the literature, to the best of our knowledge.
        \item Study of the corrections to the non-Markovianity of the dynamics, application to a solid state double quantum dot and benchmarking against exact numerical methods (HEOM) \cite{Kumar2025TCL4}.
        \item This work resulted in an open-source package that can be found on my GitHub: \\
        \href{https://github.com/prem314/spin-boson-tcl4}{\texttt{spin-boson-tcl4}}.
    \end{itemize}
    \item \textbf{Analytical extension of large coupling generalized equilibrium state to anharmonic environments:}
    \begin{itemize}
        \item Analytical derivation of the strong coupling generalized Gibbs state for a large class of anharmonic environments \cite{Kumar2024Anharm}.
        \item The harmonic environment assumption, although widely successful, is known to fail in some systems of physical relevance. 
    \end{itemize}
\end{itemize}


\sectionheading{Research Interests}

My main research interest is open quantum systems. This includes its theoretical study as well as application to physically/experimentally relevant models like those found in chemical, biological or other many-body systems. I am interested in studying both analytical and numerical techniques useful in studying these systems. I am also interested in related topics like non-Markovianity, Quantum Thermodynamics and other topics in quantum information theory.

Other than this, I am keen on exploring systems with indistinguishable particles and many body quantum systems. I am also open to exploring other interesting topics in theoretical physics.


\sectionheading{Research Statement}

I am currently focusing on expanding the scope of research in the following directions:
\begin{itemize}[leftmargin=*]
    \item Application of open quantum system techniques to study systems of physical interest. For example, I am studying the phenomenon of Chirality induced spin selectivity (CISS), specifically focusing on the temperature dependence of the phenomenon. I am also investigating some mechanism through which CISS can be observed in the absence of phonons.
    \item Higher order TCL master equation has been recently reported to have long time divergence if the bath correlator has an algebraic tail. I am performing a detailed study of the origin of this divergence using exactly solvable models and trying to find techniques to mitigate it and exploring other computationally efficient master equations that can circumvent this issue.
    \item Deriving an ultrastrong coupling master equation for anharmonic environments and studying its consistency with the corresponding generalized Gibbs state derived in my work \cite{Kumar2024Anharm}. 
    \item Recent research has revealed some general results on various constraints on the state space of a symmetric molecule with indistinguishable Fermionic nuclei. This opens up an avenue for physical effects in large molecular systems originating directly from a purely quantum mechanical effect.
    I am studying the physical implications of these results, for example in certain chemical reactions. I am also interested in the generalization of these results to more complicated molecules with higher degrees of freedom.
\end{itemize}


\sectionheading{Publications}

\nocite{Kumar2025MF,Kumar2024LHA}

\textbf{\textit{Refereed Journal Publications}}
\printbibliography[heading=none,keyword=refereed]

\textbf{\textit{Pedagogical Reviews}}
\printbibliography[heading=none,keyword=pedagogical]


\sectionheading{Technical Skills}
\begin{itemize}[leftmargin=*]
    \item \textbf{Computational Physics \& ML:} 
    \begin{itemize}
        \item \textbf{Tensor Networks:} Implementation of algorithms for simulating open quantum system dynamics using Feynman-Vernon influence functional formalism and tensor network framework.
        \item \textbf{Machine Learning:} \textbf{PyTorch} for machine learning and designing transformer models.
        \item \textbf{Languages:} Python (QuTiP, NumPy, SciPy).
    \end{itemize}
    \item \textbf{Symbolic Computing (Mathematica):} 
    \begin{itemize}
        \item Developed \href{https://github.com/prem314/spin-boson-tcl4}{\texttt{spin-boson-tcl4}}: An open-source package for the symbolic derivation and implementation of 4th-order Time-Convolutionless (TCL) master equations.
    \end{itemize}
\end{itemize}





\sectionheading{List of presentations and participations at conferences}
\begin{enumerate}
  \item Poster presentation on \textit{``Equivalence between the second order steady state for the spin-boson model and its quantum mean force Gibbs state''}, \href{https://qip2026.lu.lv/}{29th Annual Quantum Information Processing Conference (QIP 2026)}, Riga, Latvia, 24--30 January 2026.
  \item Poster presentation on \textit{``Asymptotic TCL4 Generator for the Spin-Boson Model: Analytical Derivation and Benchmarking''}, Quantum Foundations Technology and Applications (QFTA-2025), IISER Mohali, India, 4--8 December 2025.
  \item Poster presentation on \textit{``Ultrastrong coupling limit to quantum mean force Gibbs state for anharmonic environment''}, \href{https://quantum25.dpg-tagungen.de/}{Second DPG Fall Meeting: 100 Years of Quantum Physics}, Georg-August-Universität Göttingen, Germany, 8--12 September 2025.
  \item Poster presentation on \textit{``Equivalence between the second order steady state for the spin-boson model and its quantum mean force Gibbs state''}, \href{https://www.iiserkol.ac.in/~qm100/}{``100 Years of Quantum Mechanics''}, IISER Kolkata, India, 18--21 December 2024.
  \item Poster presentation on \textit{``Ultrastrong coupling limit to quantum mean force Gibbs state for anharmonic environment''}, \href{https://quantum.iitm.ac.in/qcmc24/}{QCMC-24: International Conference on Quantum Communication, Measurement and Computing}, IIT Madras, Chennai, India, 26--30 August 2024.
\end{enumerate}

\sectionheading{Talks and Seminars}
\begin{enumerate}
    \item {\it ``Quantum mean force Gibbs state in weak and ultra-strong coupling limits''}, Academic visit, ``Department of Mathematical Physics, Nicolaus Copernicus University, Torun, Poland'', Sept 15-19,
    \item Seminar on {\it ``Equilibrium state and dynamics of an open quantum system in weak and strong coupling limits''}, Academic visit, IIT-H, May 7-9 2025.
    \item Seminar on {\it ``The Approximate Thermal State for a Quantum System''}, Institute Seminar Days 2024, IMSc, Chennai, India, Jan 23 \& 31, 2024.
\end{enumerate}


\sectionheading{Achievements}
\begin{itemize}[leftmargin=*]
    \item JEST 2019: \href{https://drive.google.com/file/d/1FfxQB6itaF53i2htbFnjRr7IwUXKcZj7/view?usp=drive_link}{All India Rank 75, Percentile: 98.89}
    \item JAM 2019: \href{https://drive.google.com/file/d/1_0flgwxUQLUjuD20wbk_mOXQ7i0Sz7YF/view?usp=drive_link}{All India Rank 146, Percentile: 99.06}
    \item During BSc, won 1st prize in State level Inter-Collegiate Fest, Jyoti Nivas College, Bangalore, February 2018 for building  \href{https://www.youtube.com/watch?v=04tWPuDVBhs}{a 20-bit programmable computer on breadboard.}
\end{itemize}


\sectionheading{Professional References}

\vspace{0.5cm}
\textbf{ Dr. Sibasish Ghosh} \\
Professor \\
Optics and Quantum Information Group \\
The Institute of Mathematical Sciences \\
Chennai - 600113, India \\
e-mail: \textsl{sibasish@imsc.res.in}





\vspace{0.5cm}
\textbf{Dr. Arul Lakshminarayan} \\
Professor, Department of Physics \\
Indian Institute of Technology Madras \\
Chennai, India \\
e-mail: \textsl{arul@iitm.ac.in}


\vspace{0.5cm}
\textbf{Dr. Gniewomir Sarbicki} \\
Professor (UMK), Department of Mathematical Physics \\
Institute of Physics \\
Nicolaus Copernicus University in Torun \\
Torun, Poland \\
e-mail: \textsl{gniewko@fizyka.umk.pl}

\vspace{0.5cm}



\end{document}

